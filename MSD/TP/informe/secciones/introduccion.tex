\section{Introducción}
En este trabajo, haremos uso del modelo noruego de \citet{bjerknes1922life}
para identificar el sistema bajo estudio donde, mediante el análisis de los
mecanismos para el desarrollo de un ciclón extratropical, se plantean las
diferentes etapas de la “Ciclogénesis Clásica”. Estas etapas son:

\begin{itemize}
	\item Onda incipiente: perturbación en superficie.
	\item Autodesarrollo: Caracterizada por un aumento en la intensidad de
los sistemas en altura y en superficie.
	\item Etapa de Madurez: correspondiente al momento en el que el ciclón
	adquiere su máxima intensidad. Los cambios en el viento limitan las
	advecciones de temperatura y de vorticidad relativa.
	\item Etapa de decaimiento.
\end{itemize}

\par Continuando con esta discretización consideramos 3 tipos de ciclogénesis:
\begin{itemize}
	\item Tipo A: donde predomina el sistema frontal de superficie, se amplifica y puede
	ocurrir sin la presencia de una perturbación de onda corta en altura. Por lo
	general, los sistemas de esta categoría son poco comunes.
	\item Tipo B: se caracteriza por la presencia de una vaguada o perturbación en
	altura, que se superpone a una zona baroclínica en niveles bajos, este tipo de
	ciclogénesis es la más clásica y común.
	\item Tipo C: se caracteriza por una gran liberación de calor latente, se
	intensifican por calor diabático proveniente de la convección \citep{plant2003}.
\end{itemize}

\par En esta situación del día 28 al 30 de Abril del año 2020 analizaremos un Ciclón
de tipo C ya que ocurre una  “Ciclogénesis Explosiva” cumpliendose el criterio
definido por \citet{sanders1980synoptic}. Estos ciclones debido a su rápido
desarrollo, están asociados a fuertes vientos, rafagas y precipitaciones de
variada intensidad. Por lo tanto, el objetivo de este estudio es la
caracterización del ciclón explosivo a través de los procesos dinámicos y
termodinámicos.

\par Estudios como el de Sanders y Gyakum indican que el forzante principal es
la temperatura del mar y los contrastes de temperatura. \citet{kuo1991interaction} 
comentan que el forzante principal es el calor liberado en la convección. Por
otro lado, \cite{uccellini1984presidents} encuentran una retroalimentación entre los
flujos de la capa límite y el calor liberado. \citet{vera2002cold}, \citet{seluchi1998possible}, 
y \citet{possia2003diagnostic} argumentan que, en los casos asociados a abundante
precipitación, la liberación de calor latente por condensación es el mecanismo
más importante que intensifica el movimiento de ascenso al este del ciclón.
Luego, \citet{uccellini1990processes} indica ciertas pautas para que ocurra una ciclogénesis:
en primer lugar, debe haber una superposición de procesos dinámicos entre la
alta y baja troposfera; y en segundo lugar la interacción entre los procesos
diabáticos y dinámicos favorecen la tasa de profundización, especialmente de
aquellos procesos diabáticos relacionados con la liberación de calor latente y
aquellos que reducen la estabilidad estática en la troposfera baja.
\par De acuerdo a la climatología, estos sistemas en el sur de Sudamérica son
más frecuentes en invierno y su posición suele variar de acuerdo a la estación
según \cite{possia2004estudio}.


