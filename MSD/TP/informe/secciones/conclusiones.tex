\newpage
\section{Conclusiones}
\par Durante los días 27 al 30 de abril del 2020 se desarrolló una ciclogénesis
que inició en el norte argentino como una vaguada invertida, se
profundizó y llegó al punto de ciclogénesis explosiva que se disipó en el océano
Atlántico Sur alrededor de 41°S y 40°O.

\par Este sistema se caracterizó por contar con el aporte de flujos de calor latente y sensible una
vez que se propagó en el océano, tal como se caracterizó en estudios previos
\citep{kocin2004dynamical} donde los flujos de calor sensible y latente favorecen
el calentamiento y humedecimiento de la capa límite marina.

\par La ciudad de Mar del Plata se vió afectada desde el inicio de la formación
del sistema, a pesar de que en su etapa de desarrollo y madurez ya se encontraba
fuera del territorio.

\par El ciclón explosivo presentó una trayectoria atípica con respecto a las que son esperadas
para sistemas similares en latitudes extratropicales, lo cual puede ser un imprevisto a la hora
de pronosticar su posicionamiento y potenciales áreas de impacto.

\par Entonces se considera que es importante hacer un seguimiento en tiempo real de
este tipo de sistemas y elaborar los pronósticos adecuados para informar con
anticipación.
