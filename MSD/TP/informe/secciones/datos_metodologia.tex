\section{Datos y Metodología}
\label{sec:datos}

Se utilizaron datos observados de la estación meteorológica de Mar del Plata
Aero (código OMM: 87692), y dado que el fenómeno se desarrolló en gran parte sobre el océano
Atlántico, para analizar el entorno sinóptico se utilizaron imágenes satelitales
cada 6 horas del GOES 16 \citep{cptec_goes16}, del canal 2 (VIS), canales 8, 9 y 10 para la
visualización del vapor de agua en la atmósfera baja, media y alta (WV) y canal
13 para la visualización de topes nubosos (IR). 
\par Asumimos un sistema cuasigeostrófico para estudiar los fenómenos dinámicos y
termodinámicos que rigen en la escala sinóptica, con las siguientes ecuaciones:


\begin{equation}
\underbrace{\left( \nabla_p^2 + \frac{f_0^2}{\sigma} \frac{\partial^2}{\partial p^2} \right)\chi}_{\textbf{A}} 
= \underbrace{f_0\left[ -\vec{V}_g \cdot \nabla_p (\xi_g + f) \right]}_{\textbf{B}}
-\underbrace{\frac{f_0^2}{\sigma} \frac{\partial}{\partial p} 
\left[ \frac{R}{p}(-\vec{V}_g \cdot \nabla_p T) \right]}_{\textbf{C}}
-\underbrace{\frac{f_0^2}{\sigma} \frac{\partial}{\partial p} 
\left[ \frac{R}{p} \left( \frac{1}{c_p} \frac{dQ}{dt} \right) \right]}_{\textbf{D}}
\label{eq:qgchi}
\end{equation}

\begin{center}
\small\textbf{Ecuación \ref{eq:qgchi}.} Ecuación cuasigeostrófica de la tendencia del geopotencial (QG–$\chi$).
\end{center}

\par Se trabajó a la ecuación con el enfoque clásico:
\begin{enumerate}
    \item Se asumió que $\chi$ es una función armónica, por lo tanto el término \textbf{A} es proporcional 
    a $-\chi$
    \item El término \textbf{B}, refiere a las advecciones horizontales de vorticidad que propagan el sistema en
    el nivel isobárico que se aplique la ecuación. En este estudio, se utilizó el nivel isobárico de 300 hPa.
    \item Con respecto a \textbf{C}, término responsable de profundizar o intensificar los mínimos o máximos de altura 
    geopotencial a través de las advecciones diferenciales de temperatura entre dos niveles isobáricos, se estudió a través
    de la relación de viento térmico y la ecuación hipsométrica comparando los espesores 
    1000/500 hPa ($\phi_{500}-\phi_{1000}$).
    \item El término \textbf{D} es el que responde al calor diabático entre
    niveles, siendo de mayor importancia para este estudio aquellos intercambios
    de calor producidos por cambios de fase y flujos superficiales de calor
    sensible y latente.
\end{enumerate}


\begin{equation}
\underbrace{
\left( \nabla_p^2 + \frac{f_0^2}{\sigma} \frac{\partial^2}{\partial p^2} \right)
}_{\textbf{A}}
\omega
= \underbrace{-\frac{f_0}{\sigma} \frac{\partial}{\partial p} \left[ -\vec{V}_g \cdot \nabla_p (\xi_g + f) \right]
}_{\textbf{B}}
-
\underbrace{
\frac{R}{\sigma p} \nabla_p^2 (-\vec{V}_g \cdot \nabla_p T)
}_{\textbf{C}}
-
\underbrace{
\frac{R}{\sigma p} \nabla_p^2 \left( \frac{1}{c_p} \frac{dQ}{dt} \right)
}_{\textbf{D}}
\label{eq:omega}
\end{equation}

\begin{center}
\small\textbf{Ecuación \ref{eq:omega}.} Ecuación cuasigeostrófica del movimiento vertical en coordenadas isobáricas (QG–$\omega$).
\end{center}

\par Al igual que con la ecuación de tendencia, se enfocó el estudio de la ecuación de la siguiente manera:

\begin{enumerate}
    \item El término \textbf{A} es proporcional a $-\omega$ asumiendo que es una función armónica.
    \item El término \textbf{B} es la advección diferencial de vorticidad. En este estudio se analizó cualitativamente
    la diferencia entre las advecciones de vorticidad relativa en 300 hPa y superficie.
    \item El término \textbf{C} responde a las advecciones térmicas.
    \item Siendo \textbf{D} el término considerado para los procesos de intercambio de calor, tanto por flujos, como por
    liberación de calor latente por condensación del vapor de agua.
\end{enumerate}

\par Se realizó un análisis de dichos términos utilizando datos del modelo de reanálisis de \citet{ecmwf_era5}, 
con una resolución de $0.25^{\circ}\times0.25^{\circ}$ y 117 niveles verticales, con una altura máxima de 80 km.

\par Con este modelo generamos el espesor 1000/500 hPa para ver la evolución del
sistema en cuanto a las advecciones térmicas cálidas y frías, zonas de
baroclinicidad, y masas de aire. El nivel de 300 hPa para ver las advecciones de
vorticidad y  perturbaciones. Por otro lado, cartas de presión a nivel medio del
mar con valores de flujo de calor latente medio y calor sensible medio para
conocer el aporte de la componente de calor diabático en la intensificación del
sistema.

\par Se generó también un corte vertical en 41°S para hacer un seguimiento de las
variaciones de velocidad vertical y divergencias, donde el sistema llega a su
punto de máxima profundización.

\par Para calcular si fue una ciclogénesis explosiva, se utilizó el criterio de Sanders y Gyakum
\begin{equation}
    \left|\frac{dp}{dt}\frac{\sin{(60^{\circ})}}{\sin{(\bar{\varphi})}}\right|\geq\frac{24\mathrm{hPa}}{24\mathrm{hs}}.
    \label{eq:sanders}
\end{equation}

donde $\bar{\varphi}$ es la latitud media del mínimo de presión del sistema.

