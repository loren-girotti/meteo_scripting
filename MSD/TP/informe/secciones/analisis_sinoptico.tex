\section{Entorno Sinóptico}
\par Desde las primeras horas del día 27 en las cartas (Figura
\ref{fig:esp_vort_27-21z}) de espesores de 1000/500~hPa
se puede observar la presencia de un sistema de alta presión en el NE y centro
del territorio argentino, al mismo tiempo la formación de una vaguada invertida
en el NOA que se extiende hasta la provincia de Bs As. El avance de una vaguada
en altura con la interacción de la cordillera generó perturbaciones favoreciendo
los ascensos en niveles más bajos, lo que provocó la convección para el día 27
hacia la noche. La combinación de la vaguada invertida y del anticiclón del
Atlántico sur permitió el ingreso de aire cálido y húmedo del norte para la
formación de precipitaciones y tormentas.

\begin{figure}[htbp]
\centering
\begin{subfigure}[b]{0.9\linewidth}
    \centering
    \includegraphics[width=0.70\linewidth]{../informe/figuras/mapa_pmsl_espesor_1000_500_2020-04-27 21.png}
    \caption{Presión al nivel del mar en superficie cada 2 hPa (líneas continuas [hPa]) y espesores 1000/500 cada 20 m (sombreado [m]).}
    \label{fig:espesores_27-21z}
\end{subfigure}

\begin{subfigure}[b]{0.9\linewidth}
    \centering
    \includegraphics[width=0.70\linewidth]{../informe/figuras/mapa_geop_vort_300hPa_2020-04-27 21.png}
    \caption{Altura geopotencial en el nivel de 300hPa cada 40 m (líneas negras [m]]) y vorticidad relativa (sombreado $10^5[1/seg]$).}
    \label{fig:geop_vort_300_27-21z}
\end{subfigure}
\caption{Cartas de superficie (\ref{fig:espesores_27-21z}) y altura
(\ref{fig:geop_vort_300_27-21z}) para el día 27 de abril a las 21 UTC, generadas con el modelo ERA5.}
\label{fig:esp_vort_27-21z}

\end{figure}

\par Con el correr de las horas para el dia 28 el sistema de baja presión ya se
encontraba cerrado favorecido por los procesos dinámicos propios de la
ciclogénesis. Advecciones frías del sur por debajo de la vaguada en altura,
profundizaron a la misma, favoreciendo las advecciones de vorticidad y, en conescuencia, los
ascensos en superficie generando la caída de presión y la profundización del
sistema (se basó el análisis en los términos de las ecuaciones \ref{eq:qgchi} y
\ref{eq:omega}). Esto aumentó las advecciones de temperatura en superficie, así
como las advecciones diferenciales de temperatura. Comportamiento típico de la etapa
de autodesarrollo del modelo de ciclogénesis clásica \citep{bjerknes1922life}.

\par El día 29 el sistema de baja presión en superficie y la vaguada en la alta tropósfera
se colocaron en fase, teniendo en superficie un flujo cuasi-barotrópico que
inhibió las advecciones frías por debajo de la vaguada; a su
vez, en altura, la vaguada presentó un centro cerrado de geopotencial, producto de la profundización previa, que
impidió las advecciones de vorticidad que favorecen la inestabilidad y los
ascensos en niveles inferiores. Así, se alcanzó el fin de la etapa de autodesarrollo, para dar
comienzo a la etapa de madurez (Figura \ref{fig:esp_vort_29-6-12}).

\begin{figure}[htbp]
    \centering
    \begin{subfigure}[b]{0.9\linewidth}
        \centering
        \includegraphics[width=0.9\linewidth]{../informe/figuras/fig espesores 29-04-2025 1000_500 y 300hpa_06.png}
        \caption{Isobaras cada 2 hPa y espesores en sombreado en superficie;
        isohipsas cada 40 m y vorticidad relativa en sombreado en 300 hPa}
        \label{fig:esp_vort_29-6}
    \end{subfigure}
    \begin{subfigure}[b]{0.9\linewidth}
        \centering
        \includegraphics[width=0.9\linewidth]{../informe/figuras/fig espesores 29-04-2025 1000_500 y 300hpa_12.png}
        \caption{Isobaras cada 2 hPa y espesores en sombreado en superficie;
        isohipsas cada 40 m y vorticidad relativa en sombreado en 300 hPa}
        \label{fig:esp_vort_29-12}
    \end{subfigure}
    \caption{Análisis comparativo de los sistemas en superficie y en altura para
    el día 29/04/2020 a las 6 UTC (\ref{fig:esp_vort_29-6}) y a las 12 UTC (\ref{fig:esp_vort_29-12})}.
    \label{fig:esp_vort_29-6-12}
\end{figure}

\par Al limitarse los mecanismos dinámicos que impulsan y determinan el
desplazamiento y la profundización del ciclón, éstos se vieron dominados
por las componentes de calor sensible y latente como se puede ver en la Fig 3.
Se complementó el análisis de los flujos de calor en la región generando una imagen que muestra
las anomalías de temperatura superfcial del mar \citep{noaa_oisst_psl} para el
día 29/04/2020. Encontramos que el ciclón en superficie empieza a
propagarse hacia el NE siguiendo una trayectoria anómala respecto al patrón
típico del cinturón de los Oestes que luego de su paso por el continente, suelen tomar un
desplazamiento hacia SE por la presencia del Anticiclón Semipermanente del Atlántico Sur. Esta
propagación anómala se debe a los máximos flujos de calor (ecuación (\ref{eq:omega})) en
contraposición con el modelo clásico noruego donde indica que debería empezar a
disiparse en un caso típico de ciclones extratropicales \citep{Mendes2010_ClimatologyExtratropicalCyclones}.

\begin{figure}[htbp]
    \centering
    \begin{subfigure}[b]{0.48\linewidth}
        \centering
        \includegraphics[width=\textwidth]{../informe/figuras/mapa_pmsl_flujos_de_calor_latente_sensible_sup_2020-04-29 18.png}
        \caption{Isobaras [hPa] y flujos de calor sensible y latente [$W/m^2$]}
        \label{fig:flujo_calor}
    \end{subfigure}
    \begin{subfigure}[b]{0.48\linewidth}
        \centering
        \includegraphics[width=0.85\textwidth]{../informe/figuras/TmarAnomalia29.png}
        \caption{Anomalías de la temperatura de la superficie del mar [$^\circ C$]}
        \label{fig:anom_sst}
    \end{subfigure}
    \caption{Comparación de la anomalía de la temperatura de la superficie del mar con los flujos de calor latente y sensible,
    para el día 29/04/2020.}
    \label{fig:calor_vs_sst}
\end{figure}

\par Siguiendo el análisis, se detectó que en las anomalías de
temperatura en la superficie del mar, generado con datos de la NOAA ha mostrado
anomalías positivas que se condicen con las áreas donde se observó los máximos
flujos de calor latente y sensible, que favorecieron la convección (Ecuación (\ref{eq:omega})) y posterior condensación. Este proceso
explica que, aún sin mecanismos dinámicos que beneficien a la profundización del
ciclón, éste se benefició de los procesos termodinámicos, tal como menciona \citet{possia2011temporal}.







