\section{Resultados}

Como se mencionó en la \textbf{Sección \ref{sec:datos}}, para evaluar si se trata de una ciclogénesis
explosiva considerando el criterio de Sanders y Gyakum, utilizamos el valor de
presión del día 28 a las 18 UTC junto con el día 29 a las 18 UTC (ver cuadro \ref{tab:criterio_Sanders}). Detectando un
valor mínimo del centro de baja presión de 1002.2 hPa y 980.4 hPa,
respectivamente (Figura \ref{fig:criterio_Sanders}). Donde se estableció una latitud media de aproximadamente 41°S para la ubicación
del sistema. Cumpliendose el criterio antes mencionado (ec. \ref{eq:sanders}), un descenso de
presión mayor a 1~hPa/hs (1.22hPa/h).

\begin{table}[htbp]
    \centering
    \begin{tabular}{|l|c|c|}
        \hline
        Fecha & Presión [hPa] & Latitud [$^\circ$] \\ \hline
        28/04/2020 18 UTC & 1002.2 & -40.75 \\ \hline
        29/04/2020 18 UTC & 980.4 & -41 \\ \hline
        Resultados & $\Delta p=-21.8$ & $\bar{\varphi}=-40.84$\\ \hline 
    \end{tabular}
    \caption{Datos del mínimo de presión del ciclón estudiado para realizar el cálculo correspondiente al criterio
    de Sanders y Gyakum.}
    \label{tab:criterio_Sanders}
\end{table}

\begin{figure}[htbp]
    \centering
    \begin{subfigure}[b]{0.48\linewidth}
        \centering
        \includegraphics[width=\textwidth]{../informe/figuras/mapa_pmsl_espesor_1000_500_(minimo-marcado)_2020-04-28 18.png}
        \caption{Posición del mínimo de presión para el día 28 a las 18 UTC}
        \label{fig:criterio_Sanders_a}
    \end{subfigure}
    \begin{subfigure}[b]{0.48\linewidth}
        \centering
        \includegraphics[width=\textwidth]{../informe/figuras/mapa_pmsl_espesor_1000_500_(minimo-marcado)_2020-04-29 18.png}
        \caption{Posición del mínimo de presión para el día 29 a las 18 UTC}
        \label{fig:criterio_Sanders_b}
    \end{subfigure}
    \caption{Presión a nivel medio del mar y espesores 1000/500 hPa para los días 28 y 29 a las 18 UTC}
    \label{fig:criterio_Sanders}
\end{figure}


También se realizó la corrida de un corte vertical (Figura \ref{fig:corte_vertical}) en la latitud de 41°S para determinar el
comportamiento de los ascensos y descensos (\ref{eq:omega}) junto con los valores de
divergencia. Y se obtuvo que para el día 29 a las 12 UTC hubo fuertes valores de
convergencias y divergencias con sus respectivos valores de $\omega$ que explican
la intensificación del sistema.

\begin{figure}[htbp]
    \centering
    \includegraphics[width=0.7\linewidth]{../informe/figuras/corte_vertical_div_w_-41_2020-04-29 12.png}
    \caption{Corte vertical en 41°S para el día 29 a las 12 UTC, en sombreado
    valores de divergencias [$s^{-1}$] y en contornos $\omega$
    [$10^5\,Pa\,s^{-1}$].}
    \label{fig:corte_vertical}
\end{figure}

\subsection{Temporal en Mar Del Plata}

\par Debido al temporal asociado a esta ciclogénesis explosiva en el Sudeste de Bs As
se pudieron registrar daños estructurales, caídas de árboles y anegamientos en la localidad de Mar del Plata. 
\par Para la visualización del tiempo presente observado se utilizaron los datos del
Aeródromo de Mar del Plata. 
\par El día 27 la máxima intensidad de viento (ver figura \ref{fig:int_viento}) fue superior a 22 kt con vientos del sector este
(ver figura \ref{fig:dir_viento}), previo al pasaje del ciclón en
superficie. Luego durante el día 28, el sistema atravesó la ciudad (figura (\ref{fig:criterio_Sanders_a})),
disminuyendo la intensidad del viento debido al debilitamiento en el gradiente bárico; la dirección se hayó parcialmente
variable con predominancia del sector Norte. Finalmente, durante el día 29 se registró un aumento en la
intensidad del viento con un máximo secundario de 21 kt del sector oeste,
mientras el ciclón se situaba al este de la costa bonaerense, correspondiente a la intensificación del gradiente bárico
(figura \ref{fig:criterio_Sanders_b}). Debido a la trayectoria descripta por el ciclón, la presión a nivel medio del mar 
presentó una rápida disminución (ver figura \ref{fig:pmsnm_mdq}), la cual
comienza el día 27 alcanzado la mínima presion a la hora del paso del centro de
baja presion sobre la ciudad en el día 28 y luego para el día 29 y 30 al
retirarse el sistema hacia el este la presión se alza.
\par Por otro lado, en la serie temporal (figura \ref{fig:pp_mdq}) de precipitación acumulada cada 6 hs se
detecta un máximo de 40 mm para el día 28 finalizando con un acumulado de 74mm. Para el día previo y posterior el acumulado de
precipitación es menor a 10 mm. 

\begin{figure}[htbp]
    \centering
    \includegraphics[width=0.7\linewidth]{../informe/figuras/intensidad_viento_mdq.png}
    \caption{Intensidad del viento a 10m [kt] medido en la estación Mar del Plata Aero}
    \label{fig:int_viento}
\end{figure}

\begin{figure}[htbp]
    \centering
    \includegraphics[width=0.7\linewidth]{../informe/figuras/dir_viento_mdq.png}
    \caption{Dirección del viento a 10m [°] medido en la estación Mar del Plata Aero}
    \label{fig:dir_viento}
\end{figure}

\begin{figure}[htbp]
    \centering
    \includegraphics[width=0.7\linewidth]{../informe/figuras/pp_mdq.png}
    \caption{Precipitación acumulada cada 6hs [mm] medida en la estación Mar del Plata Aero}
    \label{fig:pp_mdq}
\end{figure}

\begin{figure}[htbp]
    \centering
    \includegraphics[width=0.7\linewidth]{../informe/figuras/pmsnm_mdq.png}
    \caption{Presión reducida al nivel medio del mar [hPa] medida en la estación Mar del Plata Aero}
    \label{fig:pmsnm_mdq}
\end{figure}


